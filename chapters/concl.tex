%! TEX root = ../root/root.tex 
\section{Conclusões}

\begin{frame}{Recaptulando}
	\begin{block}{Estratégia: Vacâncias Em Cristal Volumoso}
		\begin{itemize}
			\item v$_{\ce{Na}}^0$ insere níveis rasos doadores pois pouco distorce a rede cristalina, formação facilitada por condições oxidantes $\to$ pode potencializar a reação de evolução de oxigênio;
			\item v$_{\ce{O}}^{2+}$ insere níveis rasos próximos à banda de condução, formação facilitada por condições redutoras $\to$ pode reduzir banda proibida $\to$ absorção de luz mais próxima do visível.
		\end{itemize}
	\end{block}
	\begin{block}{Estratégia: Filme Fino}
		\begin{itemize}
		\item Em filme fino nanométrico, orientado em $[100]$, o \ce{NaNbO3} forma estados de superfície metalizados que persistem com tensão biaxial $\to$ necessária outra estratégia para produção de \ce{H2};
		\item Comparação com o \ce{NaTaO3} revela a influência da eletronegatividade: com eletronegatividade menor, \ce{Ta} forma níveis de superfície que podem participar da fotocatálise da água.
		\end{itemize}
	\end{block}
\end{frame}
\begin{frame}{Conclusões e Perspectivas}
	\begin{itemize}
		\item Bandas eletrônicas alinhadas com potenciais de oxirredução da água $+$ absorção de luz visível $+$ condução efetiva de cargas p/ sítios ativos na superfície $=$ fotocatalisador;
		\item Absorção de luz visível aumenta eficiência $+$ níveis eletrônicos das vacâncias podem diminuir a banda proibida $\to$ aumento na eficiência do \ce{NaNbO3} ortorrômbico;
		\begin{itemize}
			\item Melhor descrição dos níveis \ce{Nb} $4d$ $\to$ melhor descrição dos $\Delta E_f$ e níveis de ionização;
		\end{itemize}
		\item Filme fino de \ce{NaNbO3} orientado em $[100]$ e terminado em \ce{NaNbO} não deve ter atividade fotocatalítica $\to$ metalização persiste sob tensão biaxial;
		\item Filme fino de \ce{NaTaO3}, na mesma orientação e terminação, pode ser um fotocatalisador $\to$ níveis de superfície localizados e profundos pela eletronegatividade do \ce{Ta};
		\begin{itemize}
			\item Defeitos de superfície podem fazer os estados de superfície \ce{NaNbO3} se comportarem mais como os de \ce{NaTaO3};
		\end{itemize}
	\end{itemize}
\end{frame}
