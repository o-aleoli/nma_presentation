%! TEX root = ../root/root.tex
\section{Introdução}

\begin{frame}[allowframebreaks]{Fotocatálise Da Água}
	\begin{columns}
		\begin{column}{.5\textwidth}
			\begin{figure}[t]
				\centering
				\def\svgwidth{\linewidth}
				\input{../floats/watersplitting.pdf_tex}
				\caption{Um semicondutor precisa ter limites be banda bem alinhados com os potenciais de oxirredução da água, além de uma banda proibida que permita absorção de luz visível/UV próximo.}
			\end{figure}
		\end{column}
		\begin{column}{.5\textwidth}
			\begin{block}{Produção de \ce{H2(g)}}
				\begin{itemize}
					\item \ce{H2(g)}: Alta densidade energética, queima limpa e renovável;
					\item Pode ser produzido por quebra da molécula de água catalisada por semicondutores.
				\end{itemize}
			\end{block}
			\begin{block}{Semicondutores Ideais}
				\begin{itemize}
					\item Alinhamento dos limites da banda proibida com os potenciais de oxirredução $+$ absorção de luz visível $+$ boa condutividade $+$ sítios ativos abundantes $+$ durabilidade.
				\end{itemize}
			\end{block}
		\end{column}
	\end{columns}\framebreak
	\begin{block}{Semicondutores Reais}
		\begin{itemize}
			\item Associação à cocatalizadores, células em camadas (\textit{tandem}) $\to$ perdas de eficiência e aumento do custo produtivo;
			\item \alert{\ce{NaTaO3} e \ce{NaNbO3} são vantajosos por não precisarem de cocatalizador!}
			\begin{itemize}
				\item Porém absorvem luz no UV $\to$ atividade fotocatalítica menor que semicondutores alternativos.
			\end{itemize}
		\end{itemize}
	\end{block}
	\begin{block}{Engenharia De Bandas: Controle De Defeitos}
		\begin{itemize}
			\item Defeitos com níveis rasos, pontuais ou de superfície $\to$ aumento da condutividade, redução do tamanho de banda proibida;
			\item Vacâncias: neutras, carregadas, concentração depende das condições de síntese;
			\item Superfícies: orientação, filme fino nanométrico ou não, tensão biaxial.
		\end{itemize}
	\end{block}
\end{frame}
\begin{frame}{Objetivos}
	\begin{block}{Dos Sistemas}
		\begin{itemize}
			\item Vacâncias isoladas de \ce{Na} e \ce{O}, neutras e carregadas.
			\item Filme finos de \ce{NaNbO3} e \ce{NaTaO3} ortorrômbicos, orientados em $[100]$, terminados em \ce{NaBO} (\ce{B} $=$ \ce{Nb, Ta});
			\item Tensão biaxial nos filmes finos.
		\end{itemize}
	\end{block}
	\begin{block}{A Entender}
		\begin{itemize}
			\item Como alteram a estrutura eletrônica do cristal volumoso? Podem beneficiar a fotocatálise?
			\item Como a tensão biaxial modula a estrutura de bandas da superfície orientada?
			\item Com quais condições químicas se controla a formação de vacâncias?
			\item O filme fino de \ce{NaTaO3} pode melhorar a fotocatálise. O filme fino de \ce{NaNbO3} tem mesma potencialidade?
		\end{itemize}
	\end{block}
\end{frame}
