%! TEX root = ../root/root.tex 
\section{Metodologia e Métodos}

\begin{frame}{Métodos De Primeiros Princípios: DFT}
	\begin{figure}[t]
		\centering
		{\footnotesize%! TEX root = ../root/raiz.tex
\begin{tikzpicture}[node distance=3mm]
	\node [text width=12em, text centered] (step1) {$V_{\text{eff}} = V_{\text{ext}} + V_H[n] + V_{XC}[n]$};
	\node [below=of step1, text width=20em, text centered] (step2) {Resolver $(\hat{T} + V_{\text{eff}})\ket{\psi_i(\mathbf{r})} = \varepsilon_i \ket{\psi_i(\mathbf{r})}$ para $\varepsilon_i$};
	\node [below=of step2, text width=9em, text centered] (conv) {$\varepsilon_i - \varepsilon_{i-1} < \varepsilon_{\text{conv}}$?};
	\node [left=of conv, text width=10em] (newden) {$n_i(\mathbf{r}) = \braket{\psi_i(\mathbf{r}) | \hat{n} | \psi_i(\mathbf{r})}$};
	\node [right=of conv, text width=6em] (final) {Fim};
%%%%%%%%%%%%%%%%%%%%%%%%%%%%%%%%%%%%%%%%%%%%%%%%%%%%%%%%%%%%%%%%%%%%%%%%%%%%%%%
	\draw [arrow] (step1) -- (step2);
	\draw [arrow] (step2) -- (conv);
	\draw [arrow] (conv) -- node [anchor=west, xshift=-1.1em, yshift=1em] {não} (newden);
	\draw [arrow] (newden) |- (step1.west);
	\draw [arrow] (conv.east) -- node [anchor=east, xshift=1.1em, yshift=1em] {sim} (final.west);
\end{tikzpicture}
}
		\caption{Esquema autoconsistente da DFT para a obtenção do $\psi_0$.\label{fig:flowdft}}
	\end{figure}
	\begin{columns}
		\begin{column}{.175\textwidth}
			\includegraphics[height=50px]{../floats/logo_vasp.png}
		\end{column}
		\begin{column}{.825\textwidth}\fontsize{7}{7}
			\begin{itemize}
				\item \textit{Vienna Ab initio Simulation Package} (VASP)\footnote[frame]{\cite{kresse_efficient_1996,kresse_ultrasoft_1999}} usado com GGA-PBE;
				\item Contagem de elétrons nas vizinhanças de cada átomo com a abordagem \emph{Atoms In Molecules}\footnote[frame]{\cite{bader_quantum_1981}};
				\item Parâmetros de rede do cristal volumoso obtidos por minimização da energia total\footnote[frame]{\cite{george_effect_2018,hu_influence_2007}}:
				\begin{itemize}\footnotesize
					\item[\ce{NaNbO3}.] $a = 5.62$ ($+1.1\%$), $b = 7.90$ ($+0.9\%$) e $c = \SI{5.55}{\angstrom}$ ($+1.4\%$);
					\item[\ce{NaTaO3}.] $a = 5.62$ ($+2.0\%$), $b = 7.86$ ($+0.9\%$) e $c = \SI{5.58}{\angstrom}$ ($+1.1\%$).
				\end{itemize}
			\end{itemize}
		\end{column}
	\end{columns}
\end{frame}
\begin{frame}{Propriedades Cristalinas}
	\begin{columns}
		\begin{column}{.4\textwidth}
			\begin{figure}[t]
				\centering
				\input{../floats/bond_angles/bond_angles.pdf_tex}
				\caption{Ângulos de ligação \ce{Nb-O-Nb} $\alpha$ (direção $[\overline{1}01]$), $\beta$ (direção $[101]$), $\gamma$ (direção $[010]$).\label{fig:angles}}
			\end{figure}
		\end{column}
		\begin{column}{.5\textwidth}{\small
			\begin{itemize}
				\item Distância de ligação \ce{Nb-O} média ($\overline{d_{\ce{Nb-O}}}$) e comparação com o do cristal volumoso em estado fundamental;
				\item Ângulos entre octaedros em referência aos seus eixos, nas direções cristalinas: $[\overline{1}01]$ ($\alpha$), $[101]$ ($\beta$) e $[010]$ ($\gamma$);
				\item Distorção dos octaedros avaliada com o índice de Baur\footnote[frame]{\cite{baur_geometry_1974}}:
				\begin{equation}
					\Delta_d = \frac{1}{6} \sum^{6}_{i=1} \frac{|d_{\ce{Nb-O}\, i} - \overline{d_{\ce{Nb-O}}}|}{\overline{d_{\ce{Nb-O}}}};
				\end{equation}
				\item Filme finos: superfícies formam tetraedros \ce{NbO4}. Repulsão dos orbitais $p-d$ depende do ângulo de ligação \ce{O-Nb-O}. O índice de planaridade $P_L$ descreve quâo próximo está do ideal:
				\begin{equation}
					P_L = \frac{\theta_{\ce{O-Nb-O}}}{\SI{180}{\degree}};
				\end{equation}
			\end{itemize}}
		\end{column}
	\end{columns}
\end{frame}
\begin{frame}{Estimativa Da Energia De Formação De Vacância}
	\begin{itemize}{\small%
		\item A energia de formação $\Delta E_f$ para vacâncias v$_{\ce{X}}^{q} = \text{v}_{\ce{O}}^{0}, \text{v}_{\ce{O}}^{2+}, \text{v}_{\ce{Na}}^{0}, \text{v}_{\ce{Na}}^{1-}$:
		\begin{equation}
			\Delta E_f (\text{v}_{\ce{X}}^{q}; L) =
			\tikz[baseline] {\node[anchor=base, fill=red!50] (part1) {$\!E_T (\text{v}_{\ce{X}}^{q}; L)\!$};} -
			\tikz[baseline] {\node[anchor=base, fill=green!40] (part2) {$\!E_T (\text{perfeito}; L)\!$};} -
			\tikz[baseline] {\node[anchor=base, fill=blue!50] (part3) {$\!n_{\ce{X}} \mu_{\ce{X}}\!$};} +
			\tikz[baseline] {\node[anchor=base, fill=violet!50] (part4) {$\!q \varepsilon_F\!$};} +
			\tikz[baseline] {\node[anchor=base, fill=black!40] (part5) {$\!E_c$};}.
		\end{equation}
		\begin{itemize}
			\footnotesize
			\item\tikz{\node [fill=red!50] (description1) {Energia total da supercélula com v${}_{\ce{X}}^{q}$;};}
			\item\tikz{\node [fill=green!40] (description2) {Energia total do cristal volumoso sem defeitos;};}
			\item\tikz{\node [fill=blue!50] (description3) {Troca com reservatório de \ce{X};};}
			\item\tikz{\node [fill=violet!50] (description4) {Troca com reservatório de e${}^{-}$;};}
			\item\tikz{\node [fill=black!40] (description5) {Correção de preenchimento de banda.};}
		\end{itemize}
		\item Interações vacância-imagens $\to$ imprecisão na energia de formação $\propto L = V^{{}^{1}\!/{}_{3}}$;
		\item Extrapolação $L\to\infty$ (limite diluído) modelada com\footnote[frame]{\cite{castleton_managing_2006}}:
		\begin{equation}
			\Delta E_f (\text{v}_{\ce{X}}^{q}; L) = \Delta E_f (\text{v}_{\ce{X}}^{q}; L\to\infty) + a_1 L^{-1} + a_3 L^{-3}.
		\end{equation}
		\begin{itemize}
			\item Supercélulas repetem a célula convencional nos eixos cartesianos: $1\!\times\!1\!\times\!1$, $2\!\times\!2\!\times\!2$ e $3\!\times\!3\!\times\!3$;
			\item Parâmetros $\Delta E_f (\text{v}_{\ce{X}}^{q}; L\to\infty)$, $a_1$ e $a_3$ determinados pelo método dos mínimos quadrados.
		\end{itemize}
		}
	\end{itemize}
\end{frame}
\begin{frame}{Estimativa Da Energia De Formação De Vacância: Potencial Químico}
	\begin{columns}
		\begin{column}{.45\textwidth}
			\begin{figure}[t]
				\centering
				\begin{tikzpicture}
        \begin{ternaryaxis}[
        width=\textwidth,
        xlabel=O ($\%$),
        xlabel style={%
                at={(axis cs:1,0,0)},
                anchor=south
        },
        xtick={0.2,0.4,0.6,0.8},
        xticklabels={$20$,$40$,$60$,$80$},
        xticklabel style={font=\footnotesize},
        xtick style={line width=0.25mm, color=black},
        ylabel=Na ($\%$),
        ylabel style={%
                at={(axis cs:0,1,0)},
                anchor=north,
                rotate=-60
        },
        ytick={0.2,0.4,0.6,0.8},
        yticklabels={$20$,$40$,$60$,$80$},
        yticklabel style={font=\footnotesize},
        ytick style={line width=0.25mm, color=black, font=\tiny},
        zlabel=Nb ($\%$),
        zlabel style={%
                at={(axis cs:0,0,1)},
                anchor=north,
                rotate=60
        },
        ztick={0.2,0.4,0.6,0.8},
        zticklabels={$20$,$40$,$60$,$80$},
        zticklabel style={font=\footnotesize},
        ztick style={line width=0.25mm, color=black, font=\tiny},
        clip=false,
        no markers,
        grid=none
        ]
                \addplot3 [color=black] table {%
                        O        Na       Nb 
                        1.0      0.0      0.0
                        0.6      0.2      0.2
                        0.666667 0.333333 0.0
                };
            
                \addplot3 [color=black] table {%
                        O        Na       Nb 
                        0.666667 0.333333 0.0
                        0.6      0.2      0.2
                        0.5      0.5      0.0
                };
            
                \addplot3 [color=black] table {%
                        O        Na       Nb 
                        0.5      0.5      0.0
                        0.6      0.2      0.2
                        0.333333 0.666667 0.0
                };
            
                \addplot3 [color=black] table {%
                        O        Na       Nb 
                        0.333333 0.666667 0.0
                        0.6      0.2      0.2
                        0.0      1.0      0.0
                };
            
                \addplot3 [color=black] table {%
                        O        Na       Nb 
                        0.0      1.0      0.0
                        0.6      0.2      0.2
                        0.0      0.0      1.0
                };
            
                \addplot3 [color=black] table {%
                        O        Na       Nb 
                        0.0      0.0      1.0
                        0.6      0.2      0.2
                        0.5      0.0      0.5
                };
            
                \addplot3 [color=black] table {%
                        O        Na       Nb      
                        0.5      0.0      0.5     
                        0.6      0.2      0.2     
                        0.666667 0.0      0.333333
                };
            
                \addplot3 [color=black] table {%
                        O        Na       Nb      
                        0.714286 0.0      0.285714
                        0.6      0.2      0.2     
                        0.666667 0.0      0.333333
                };
            
                \addplot3 [color=black] table {%
                        O        Na       Nb      
                        0.714286 0.0      0.285714
                        0.6      0.2      0.2     
                        1.0      0.0      0.0     
                };
            
                \node [circle, draw, fill=red, pin={[pin distance=3ex]270:\ce{NaNbO3}}] at (axis cs:0.6,0.2,0.2) {};
                \node [pin={[pin distance=5ex]120:\ce{NaO2}}] at (axis cs:0.666667, 0.333333, 0.0) {};
                \node [pin={[pin distance=5ex]150:\ce{Na2O2}}] at (axis cs:0.5, 0.5, 0.0) {};
                \node [pin={[pin distance=5ex]150:\ce{Na2O}}] at (axis cs:0.333333, 0.666667, 0.0) {};
                \node [pin={[pin distance=5ex]0:\ce{NbO}}] at (axis cs:0.5, 0.0, 0.5) {};
                \node [pin={[pin distance=5ex]0:\ce{NbO2}}] at (axis cs:0.666667, 0.0, 0.333333) {};
                \node [pin={[pin distance=5ex]20:\ce{Nb2O5}}] at (axis cs:0.714286, 0.0, 0.285714) {};
        
                \node [] at (cartesian cs:0.45, 0.65) {{\footnotesize A}};
                \node [] at (cartesian cs:0.35, 0.51) {{\footnotesize B}};
                \node [] at (cartesian cs:0.275, 0.4) {{\footnotesize C}};
                \node [] at (cartesian cs:0.2, 0.265) {{\footnotesize D}};
                \node [] at (cartesian cs:0.5, 0.175) {{\footnotesize E}};
                \node [] at (cartesian cs:0.71, 0.375) {{\footnotesize F}};
                \node [] at (cartesian cs:0.65, 0.51) {{\footnotesize G}};
                \node [pin={[pin distance=4ex, pin edge={black, thick}]7.5:{\footnotesize H}}] at (cartesian cs:0.6, 0.57) {};
                \node [] at (cartesian cs:0.55, 0.65) {{\footnotesize I}};
        \end{ternaryaxis}
\end{tikzpicture}
				\caption{Diagrama ternário do \ce{NaNbO3} mostrando as regiões onde se avaliou $\mu_{\ce{X}}$ de v$_{\ce{X}}^q$.\label{fig:ternarypd}}
			\end{figure}
		\end{column}
		\begin{column}{.55\textwidth}
			\begin{itemize}
				\item Equilíbrio químico com o \ce{NaNbO3}: $\mu^{0}_{\ce{NaNbO3}} = \mu_{\ce{Na}} + \mu_{\ce{Nb}} + 3 \mu_{\ce{O}}$;
				\item Condições de síntese mudam $\mu_{\ce{X}}$, variando dentro dos limites impostos pelos reagentes e.g. no equilíbrio I:
				\begin{equation*}
					\begin{cases}
						2 \mu_{\ce{O}} = \mu_{\ce{O2}}^{0};\\
						2 \mu_{\ce{Nb}} + 5 \mu_{\ce{O}} = \mu_{\ce{Nb2O5}}^{0};\\
						\mu_{\ce{Na}} + \mu_{\ce{Nb}} + 3 \mu_{\ce{O}} = \mu_{\ce{NaNbO3}}^{0}.
					\end{cases} \Rightarrow
				\end{equation*}
				\begin{equation}
					\begin{cases}
						\mu_{\ce{O}} = {}^{1}\!/{}_{2}\, \mu_{\ce{O2}}^{0};\\
						\mu_{\ce{Nb}} = {}^{1}\!/{}_{2}\, \mu_{\ce{Nb2O5}}^{0} + {}^{5}\!/{}_{4}\, \mu_{\ce{Nb2O5}}^{0};\\
						\mu_{\ce{Na}} = \mu_{\ce{NaNbO3}}^{0} - {}^{1}\!/{}_{2}\, \mu_{\ce{Nb2O5}}^{0} - {}^{1}\!/{}_{4}\, \mu_{\ce{O2}}^{0}.
					\end{cases}
				\end{equation}
			\end{itemize}
		\end{column}
	\end{columns}
\end{frame}
\begin{frame}{Estudo Dos Filmes Finos}
	\begin{columns}
		\begin{column}{.5\textwidth}
			\begin{figure}[t]
				\centering
				\input{../floats/slabs/slabs.pdf_tex}
				\caption{Filme fino de \ce{NaNbO3}. Filme de \ce{NaTaO3} construído de maneira análoga.\label{fig:slabs}}
			\end{figure}
		\end{column}
		\begin{column}{.5\textwidth}
			\begin{itemize}
				\item Supercélula: célula convencional $\times 3$ em $[100]$ $+$ vácuo de \SI{15}{\angstrom};
				\begin{itemize}
					\item 9 camadas atômicas, 86 átomos.
				\end{itemize}
				\item Orientação da clivagem em $[100]$ e terminação \ce{NaNbO} $\to$ tetraedros \ce{NbO4};
				\item Remoção de duas ligações \ce{Nb-O} $\to$ carga formal $\to$ reconstrução de superfície;
				\item Estado fundamental descrito com \alert{$b$ e $c$ $1\%$ maiores que do cristal volumoso} $\to$ referência para tensão biaxial.
				\item Filme de \ce{NaTaO3} construído da mesma forma, estado fundamental com $b$ e $c$ idênticos ao do cristal volumoso. Uso em fotocatálise discutido em outro trabalho\footnote[frame]{\cite{kraus_modulation_2020}}.
			\end{itemize}
		\end{column}
	\end{columns}
	\begin{center}
		Figuras da dissertação no \href{https://github.com/o-aleoli/nma_dissertation_fig}{\color{blue}{repositório do GitHub}}.
	\end{center}
\end{frame}
